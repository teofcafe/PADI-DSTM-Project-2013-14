%
\documentclass[times, 10pt,twocolumn]{article} 
\usepackage{latex8}
\usepackage{times}
\usepackage[portuguese]{babel}
\usepackage[utf8]{inputenc}
\usepackage{graphicx}
%\documentstyle[times,art10,twocolumn,latex8]{article}

\pagestyle{empty}

\begin{document}

\title{\huge{PADI-DSTM} \\[0,1in] \textmd{Plataformas para Aplicações Distribuídas na Internet \\[0,05in] 2013-14}}

\maketitle
\thispagestyle{empty}

\begin{abstract}
O PADI-DSTM é um sistema distribuído que tem como objectivo gerir objectos alojados em diferentes servidores, com recurso a transacções. A solução apresentada segue a abordagem de TimeStamp Ordering[1], em que o master faculta timestamps únicos para cada transacção. Os vários servidores, além de alojarem e manipularem dados, também podem ser coordenadores transaccionais. O cliente sabe quem é o coordenador das suas transacções através do master. Esta solução já inclui replicação de dados, para manter a redundância dos mesmos.
\end{abstract}

\section{Introdução}

O número crescente de dispositivos e necessidades distintas no contexto da computação levou à evolução do paradigma da oferta de serviços. Podemos destacar as soluções associadas a sistemas distribuídos, devido à escalabilidade e paralelismo que oferecem na execução das tarefas. Apesar das vantagens associadas a este tipo de soluções, existem também grandes desafios quando se opta por uma abordagem distribuída. O \textbf{PADI-DSTM} tenta superar esses desafios, tirando o máximo de partido da tecnologia.



\section{Arquitectura}

\subsection{Componentes do Sistema}

\subsubsection{\textit{Master}}
O \textit{master} é um componente fundamental do nosso sistema, uma vez os pedidos dos clientes têm como ponto de referencia  inicial o \textit{master}.
\begin{description}

\item[Criação de transacções: ]
 Quando um cliente deseja iniciar uma transacção, conecta-se ao \textit{master} (cujo endereço é previamente conhecido por todos os clientes). Quando este é criado, regista o seu canal TCP e fica à escuta de pedidos dos clientes. Estes após se ligarem ao \textit{master}, o mesmo cria uma transacção passando pelas seguintes fases:
\begin{itemize}

\item \textbf{Geração do \textit{timestamp}:} De modo a  garantir que o sistema opera sobre \textit{timestamps} sincronizados, optámos por usar o \textit{master} como gerador de \textit{timestamps}, uma vez que é único no nosso sistema. O processo de geração de \textit{timestamps} consiste em consultar o seu \textit{Real Time Clock} (RTC) com a precisão de microssegundos. Este processo de entrega de \textit{timestamps} pelo \textit{master} faz com que todos os clientes criem transacções com \textit{timestamps} sequenciais. Isto é uma vantagem, uma vez que faculta uma ordem de acesso universal para o caso de diferentes transacções acederem ao mesmo objecto. A consulta do RTC não tem um peso computacional considerável, sendo esta, outra vantagem.

\item \textbf{Escolha do coordenador:} O \textit{master} tem conhecimento de todos os servidores existentes no sistema, devido ao registo destes últimos. As URLs dos servidores existentes são guardadas em dois tipos de lista: servidores disponíveis e servidores sobrecarregados. A vantagem desta abordagem resume-se no facto do \textit{master} ter  sempre conhecimento quando um servidor está sobrecarregado, podendo tomar as acções apropriadas para manter o sistema equilibrado.

Para a escolha do coordenador, o \textit{master} selecciona um servidor aleatoriamente da lista de servidores disponíveis. O facto da escolha do coordenador ser feita de modo aleatório traz vantagens, uma vez que é mais rápida e a probabilidade do mesmo servidor ser escolhido duas vezes seguidas diminui conforme o aumento do número de servidores no sistema. Além disto, esta abordagem permite que o trabalho de coordenar as transacções seja poupado ao \textit{master}, sendo feito pelos coordenadores que não são mais que servidores.

Antes do servidor ser eleito coordenador da transacção, o \textit{master} verifica se esse servidor entretanto ficou sobrecarregado. Em caso afirmativo, o \textit{master} adiciona-o à lista de servidores sobrecarregados e escolhe outro servidor, que passa pelo mesmo processo. Caso o servidor escolhido não esteja sobrecarregado, ficará encarregue de coordenar aquela transacção.  Esta abordagem permite abstrair o cliente da detecção da sobrecarga de cada servidor e da sua respectiva selecção. Retira também parte dessa mesma carga dos servidores, de modo a que o poder computacional destes últimos seja o mais possível focado na execução das transacções, para minimizar ao máximo o número de transações que abortam, uma vez que a nossa abordagem é de \textit{TimeStamp Ordering}\cite{ex1}.
\end{itemize}
Após a geração/escolha destes dois parâmetros, o \textit{master} cria a transacção e devolve-a ao cliente e volta a ficar à escuta de novos pedidos.

\item[Registo de novos servidores e localização de objectos:]
O \textit{master} decide em que servidor vão ser criados os novos objectos resultantes dos pedidos dos clientes. A decisão é baseada numa fórmula que tenta dividir ao máximo, de forma equivalente, a carga total de armazenamento pelos servidores. Deste modo, o ID do servidor em que vai ser criado o novo objecto é igual ao resultado de $ \textit{hash(uid) mod n} $, onde n corresponde ao número de servidores e uid ao identificador numérico do objecto. Para guardar o objecto primário, sem ter em conta a réplica, a função de \textit{hash(uid)} devolve o próprio uid. Esta fórmula serve também para aceder a objectos já existentes, e nesse caso o uid corresponde ao uid do objecto a que se quer aceder.

\item[Objectos especiais:] 
A fórmula anterior é válida para todos os objectos, excepto para aqueles são extremamente acedidos e residem no mesmo servidor. Estes são designados por objectos especiais. Para estes objectos é guardado no \textit{master} a sua localização real, visto que a fórmula não iria resultar neste caso. 
Deste modo, após o coordenador executar a fórmula para saber onde está o objecto, caso o retorno seja \textit{null}, então é porque esse objecto é um candidato a objecto especial (isto porque o objecto pode nem sequer existir). Face a esta situação, então é pedido ao \textit{master} a localização daquele potencial objecto especial. Caso o coordenador não consiga aceder ao servidor proveniente da resposta relativa a esse objecto especial, então é porque o servidor responsável pela recepção do objecto especial está indisponível. Devido a isto, em ultimo caso, o coordenador pede ao \textit{master} a localização da réplica deste objecto especial. O facto de ser o coordenador a fazer as tentativas de acesso ao invés de ser o \textit{master} a verificar se o servidor que vai na resposta está indisponível ou não é uma mais valia, na medida em que esta abordagem optimista na devolução dos servidores (responsáveis por um dado uid especial) pelo \textit{master} evita verificações que podiam ser desnecessárias. Além de evitarmos trabalho computacional desnecessário, distribuímos o necessário pelos vários coordenadores, ao invés de fazermos as verificações no \textit{master}, que é único.

\item[Actualização dos servidores pelo master:] 
Sempre que um servidor é criado, o \textit{master} fornece-lhe a informação relativa a esta fórmula. Do mesmo modo, cada vez que um novo servidor aparece no sistema ou um objecto é considerado especial essa informação tem que ser actualizada. Deste modo, à medida que os servidores tentam aceder à informação não actualizada é apresentado um erro. Quando os servidores se deparam com este erro, requisitam a informação ao \textit{master} e nesse momento têm acesso à informação actualizada. A partir do momento que o servidor se regista no \textit{master}, pode também ser escolhido para ser coordenador, visto que a selecção se dá de forma aleatória entre todos os servidores disponíveis. Para facilitar o trabalho do \textit{master}, este guarda na tabela a correspondência entre o ID dos servidores e os respectivos endereços IP. 

\begin{table}[htb]
\centering
\begin{tabular}{c|c}
ID do Servidor & IP do Servidor \\\hline
0 & tcp : // 192.168.2.72:80 /Server \\
1 & tcp : // 192.168.2.73:80 /Server \\
2 & tcp : // 192.168.2.74:80 /Server \\
3 & tcp : // 192.168.2.75:80 /Server \\
\end{tabular}
\caption{\label{tab:idip}Tradução ID/IP dos servidores disponíveis.}
\end{table}

Parte desta informação é enviada para o coordenador caso o mesmo faça um pedido para tal. De modo a diminuir o número de pedidos, cada coordenador tem uma cache para este efeito.

\end{description}

\subsubsection{Servidor}

Os servidores têm como função guardar e alterar os objectos partilhados, deste modo podem ser vistos como um repositório. Têm responsabilidade de se registar no mestre assim que estejam activos. Quando um objecto é alterado com sucesso, é da responsabilidade do servidor encaminhar essa informação até ao coordenador dessa transacção. O servidor tem disponíveis métodos de criação, leitura e escrita de objectos. O servidor guarda os objectos num vector, onde o índice do vector é o id do objecto. 

Existe uma \textit{thread} de baixa prioridade que de 10 em 10 segundos verifica o peso do próprio servidor. Se esse peso atingir um máximo definido (por optimização, é um valor variável) o servidor avisa o \textit{master} que está sobrecarregado. Cada vez que é verificado o peso do servidor, essa variável é colocada a 0. 

Existe também outra \textit{thread} de baixa prioridade que corre no fim de ter ocorrido uma restruturação dos dados num servidor devido à entrada de um novo servidor, e que é responsável por marcar os dados que serão apagados ou migrados quando entrar um novo servidor, de modo a que a entrada de um novo servidor termine o mais rapidamente possível. Cada PadInt contém um enum\{DELETE, MIGRATE, NONE\}, que é alterado pela \textit{thread} responsável pela marcação. Deste modo, prevê-se sempre que os dados serão migrados para um novo servidor, caso venha a aparecer. Isto garante que a fórmula utilizada para encontrar os objectos nos vários servidores funciona sempre, mesmo que o número de servidores varie ao longo do tempo.

O tipo PadInt é composto, além do seu valor numérico, por um \textit{mutex} e por um contador de acessos e pelo \textit{timestamp} da última versão que fez \textit{commit} com sucesso. Este objecto é guardado num vector, em que o índice é a posição correspondente ao uid desse objecto. Cada servidor guarda uma tabela de dispersão criada especificamente para gerir intervalos de ids onde guarda todos os objectos que lhe foram atribuídos.

Cada coordenador transaccional gere ainda um conjunto de tentativas transaccionais para quando tentar fazer \textit{commit} ter essa informação disponível. Esta informação é também útil caso outras transacções necessitem dos dados que ainda não tenham feito \textit{commit} (para saberem que têm que esperar ao invés de abortar). 

Para o servidor detectar os objectos especiais (que têm uma elevada taxa de acesso) usa um critério relacionado com a um número considerado limite máximo de acessos a um objecto. Este valor é variável conforme os testes feitos ao sistema e optimizado para o melhor desempenho possível. Se dois objectos num determinado servidor excederem esse valor, considera-se que os dois objectos em questão são especiais. Quando isto acontece, o \textit{master} é notificado para proceder ao balanceamento de carga por outros servidores. Isto significa que um desses objectos é escolhido aleatoriamente para ser mudado de servidor. Os servidores verificam esta situação em cada acesso (\textit{read} ou \textit{write}) e caso tal se verifique, a situação é reportada ao \textit{master}.


\subsubsection{Cliente}

Os clientes são os utilizadores do serviço. Cada cliente conhece, inicialmente, apenas o endereço do master. Se o cliente optar por criar um novo objecto, esse pedido segue directamente para o master que dá continuidade ao resto do processo. Se o cliente optar por aceder a um objecto que já existe, esse pedido segue para o master, que devolve o endereço do servidor que contém esse objecto e adicionalmente os ids do resto dos objectos alocados nesse mesmo servidor. O resto do processo resume-se na interacção cliente-servidor, sendo que o master só interfere na coordenação transaccional. Tal como já foi referido, este processo pode reduzir o número de pedidos feitos ao master, o que melhora a performance dos serviços. Durante este processo o cliente pode ser avisado se algo correr mal, ou então é avisado no fim que tudo correu bem. 

\subsubsection{Transac��es e Controlo de Concorr�ncia}

Quando o cliente faz um pedido ao master, seja de cria��o de novo objectos ou acesso a objectos j� existentes, o master inicia uma nova transac��o atrav�s do m�todo TxBegin(). Ap�s isso, os pedidos s�o enviados para os servidores, j� inseridos naquela transac��o espec�fica. Durante o processamento, caso existam conflitos � retornado um valor boolean negativo, lan�ada a exep��o TxException e a opera��o aborta atrav�s do m�todo TxAbort(). 

Para o controlo de concorr�ncia � usada a abordagem TimeStamp Ordering. Ao iniciar, as transac��es ganham um timestamp. Isto permite impor a ordem de acesso aos dados e consequentemente resolver os conflitos mais cedo. Deste modo, caso uma transac��o queira escrever num objecto (vari�vel) que j� lido por uma transac��o com timestamp superior ao seu, detecta-se logo que isto n�o � v�lido, porque n�o mantinha a coer�ncia dos dados. Assim que esta anomalia � detectada � feito abort e reduz-se a quantidade de trabalho desperdi�ado, devido � detec��o precoce. O mesmo se aplica caso uma transac��o queira escrever ou ler num objecto que foi escrito por uma transac��o com timestamp superior. Tome-se por excep��o o caso em que uma transac��o de escrita tenta escrever num objecto que j� foi escrito por uma transac��o com timestamp superior, n�o tendo este objecto ainda ter sido lido por outra transac��o. Neste caso a escrita atrasada pode ser ignorada ao inv�s de ser feito abort, visto que ia ser sobreposta. Isto pode poupar recursos na medida em que se diminuem o n�mero de transac��es que reiniciam ap�s terem sido abortadas.

Outra vantagem desta abordagem � que n�o existem deadlocks. Al�m disto, quando uma transac��o chega ao fim sabemos que pode fazer commit, pois caso existisse alguma anomalia j� tinha sido detectada anteriormente. Caso existam no log vers�es a commitar e que a transac��o actual precisa de ler, a abordagem escolhida � esperar pelo commit da outra transac��o para prosseguir. Apesar da espera isto ser� transparente para o cliente.


\input{Functionality}



\section{Conclusão}

A solução proposta de modo a tornar o \textbf{PADI-DSTM} uma aplicação sequencialmente consistente reside numa abordagem de TimeStamp Ordering\cite{ex1}. Nesta solução já estão contempladas tanto a tolerância a faltas como a migração de objectos extremamente acedidos, existindo distribuição de carga. Isto permite que o sistema se mantenha equilibrado, tanto a nível de computação como a nível de armazenamento. Parte das decisões tomadas para esta arquitectura (por exemplo, métricas que consideram um PadInt extremamente acedido), podem ser adaptados aos mais variados tipos de sistema, de modo a maximizar o desempenho. Analisando os resultados experimentais dos testes feitos ao nosso sistema, usando processadores intel i7, verificamos que o mesmo se mantém escalável e consistente, mesmo quando sobrecarregado. 


\nocite{ex1}
\bibliographystyle{latex8}
\bibliography{latex8}

\end{document}