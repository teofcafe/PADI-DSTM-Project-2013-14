\subsubsection{Servidor}

\begin{description}

\item[Gestão e manipulação de dados:]
Os servidores têm como função guardar e alterar os objectos partilhados, deste modo podem ser vistos como um repositório. Têm responsabilidade de se registar no mestre assim que estejam activos. Quando um objecto é alterado com sucesso, é da responsabilidade do servidor encaminhar essa informação até ao coordenador dessa transacção. O servidor tem disponíveis métodos de criação, leitura e escrita de objectos. O tipo PadInt é composto, além do seu valor numérico, por um contador de acessos, pelo \textit{timestamp} da última versão que fez \textit{commit} com sucesso e por um \textit{enum\{DELETE, MIGRATE, NONE\}}. Cada objecto é guardado numa  tabela de dispersão criada especificamente para gerir intervalos de uids. Cada uid serve de \textit{key} para aceder ao objecto na tabela.

\item[Carga dos servidores:]
Existe uma \textit{thread} de baixa prioridade que, de 10 em 10 segundos, verifica o peso do próprio servidor. Se esse peso atingir um máximo definido (por optimização, é um valor variável conforme o servidor) o servidor avisa o \textit{master} que está sobrecarregado. Cada vez que é verificado o peso do servidor, essa variável é colocada a 0.

\item[Marcação de dados a migrar:]
Existe também outra \textit{thread}, de baixa prioridade, que corre no fim de ter ocorrido uma restruturação dos dados num servidor (devido à entrada de um novo servidor), e que é responsável por marcar os dados que serão apagados ou migrados quando entrar novamente um servidor. Este processo permite a que a entrada de um novo servidor ocorra o mais rapidamente possível. O \textit{enum\{DELETE, MIGRATE, NONE\}} do objecto em questão, é alterado pela \textit{thread} responsável pela marcação. Deste modo, prevê-se sempre que dados serão migrados para um novo servidor, caso este venha a aparecer. Isto garante que a fórmula utilizada para encontrar os objectos nos vários servidores funciona sempre, mesmo que o número de servidores varie ao longo do tempo.

\item[Versões tentativas:]
Cada coordenador transaccional gere ainda um conjunto de tentativas transaccionais (versões tentativas dos dados antes do \textit{commit}), para quando tentar fazer \textit{commit} ter essa informação disponível. Esta informação é também útil caso outras transacções necessitem dos dados que ainda não tenham feito \textit{commit} (para saberem que têm que esperar ao invés de abortar).

\item[Especificação de objectos especiais:]
Para o servidor detectar os objectos especiais (que têm uma elevada taxa de acesso) usa um critério relacionado com a um número considerado limite máximo de acessos a um objecto. Este valor é variável conforme os testes feitos ao sistema e optimizado para o melhor desempenho possível. Se dois objectos num determinado servidor excederem esse valor, considera-se que os dois objectos em questão são especiais. Quando isto acontece, o servidor notifica o \textit{master}, para que este proceda ao balanceamento de carga por outros servidores. Isto significa que um desses objectos é escolhido aleatoriamente para mudar de servidor. Os servidores verificam esta situação em cada acesso (\textit{read} ou \textit{write}).

\end{description}