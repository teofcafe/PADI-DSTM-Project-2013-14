\subsubsection{Master centralizado}

O master centralizado tem como fun��o coordenar as transac��es. Deste modo mant�m um log das altera��es feitas aos objectos, permitindo �s que s�o consideradas correctas na altura da valida��o fa�am commit, dando ordem de abort em caso contr�rio. Caso uma transac��o chegue � fase de commit, e devido � abordagem utilizada, o cliente � logo informado que o pedido foi executado com sucesso. Guarda ainda informa��o sobre os servidores activos na actualidade, permitindo que estes, quando arrancam, se registem nele. Quando um servidor deixa de estar activo, a lista de servidores do mestre � actualizada. O registo � feito numa estrutura auxiliar, um array de string que guarda o URL dos novos servidores. Mant�m ainda, informa��o sobre os objectos que residem em cada servidor.

O modo como a informa��o referente aos servidores e aos objectos que os mesmos guardam est� guardada numa tabela de dispers�o. O identificador da tabela � o id do objecto guardado e a cada objecto est� associado um servidor onde o mesmo reside.

\begin{table}
\centering
\begin{tabular}{c|c}
ID do Objecto & IP do Servidor \\\hline
1 & tcp : // 192.168.2.72:80 /Server \\
2 & tcp : // 192.168.2.73:80 /Server \\
3 & tcp : // 192.168.2.74:80 /Server \\
4 & tcp : // 192.168.2.75:80 /Server \\
\end{tabular}
\caption{\label{tab:widgets}Tabela de dispers�o do master.}
\end{table}

Cada vez que um cliente faz um pedido de acesso a um objecto (identificado por um id num�rico), o master devolve o URL no formato ?tcp : // <ip-address>:<port> /Server? do servidor que cont�m esse objecto, bem como, os restantes objectos presentes no reposit�rio desse mesmo servidor.  Isto � vantajoso, porque permite que o cliente tenha conhecimento de todos os objectos presentes no servidor em quest�o, de modo a que, se eventualmente, queira alterar um deles, n�o tenha que fazer um novo pedido. Caso o cliente fa�a um pedido de um objecto que n�o exista �-lhe devolvido null. Caso o cliente queira criar um novo objecto, o master consulta o array de novos servidores para verificar se existe algum servidor sem objectos ou n�o, e caso exista, d� ordem ao servidor para que crie o objecto no seu reposit�rio e de seguida, apaga esse endere�o do array de novos servidores e adiciona o URL do mesmo servidor � tabela de dispers�o juntamente com o id do objecto. 