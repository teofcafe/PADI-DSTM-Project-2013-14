\subsection{Replicação}

A fórmula para encontrar o servidor em que está localizado um objecto com um dado id é \textit{"(hash(id)) mod (número de servidores)"}. Para encontrar o servidor com o objecto primário a função de dispersão retorna o id, para encontrar o servidor com a réplica a função de dispersão retorna id + j (número de objectos por bloco).
O \textit{master} contém as fórmulas mais recentes e tanto os servidores como os clientes guardam estas fórmulas em \textit{cache} no primeiro pedido efectuado ao master, para evitar pedidos remotos desnecessários.

Os clientes podem aceder aos dados de 2 maneiras:

\begin{itemize}
\item \textbf{Com replicação:} 

\begin{enumerate}
\item Cada leitura é efectuada ao servidor primário do objecto, e caso este não esteja disponível o valor é obtido do servidor secundário.

\item Cada escrita é efectuada no primário e no secundário, mas o cliente não espera pelas respostas. Caso existam mais de 2 réplicas, o cliente envia no máximo a 5 que replicam para as que faltam.
\end{enumerate}

\item \textbf{Sem replicação:}

\begin{enumerate}
\item Cada leitura é efectuada apenas no servidor responsável por guardar aquele objecto, se o servidor estiver em baixo a transacção é abortada.

\item Cada escrita é efectuada apenas no servidor responsável por guardar aquele objecto, se o servidor estiver em baixo a transacção é abortada.
\end{enumerate}
\end{itemize}