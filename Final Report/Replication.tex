\subsubsection{Replicação}
A fórmula para encontrar o servidor em que está localizado um objecto com um dado uid é: $\textit{hash(uid) mod n}$. Para encontrar o servidor com o objecto primário a função de dispersão retorna o uid, para encontrar o servidor com a réplica a função de dispersão retorna uid+1. O \textit{master} contém as fórmulas mais recentes e os servidores/coordenadores guardam estas fórmulas em cache no primeiro pedido efectuado ao \textit{master}, para evitar pedidos remotos desnecessários.
As operações efectuadas pelos clientes quando acedem a um dado objecto podem ter os seguintes rumos:

\begin{enumerate}
\item
Cada leitura é efectuada no servidor primário e secundário do objecto, e caso o primário não esteja disponível o valor é obtido do secundário.
\item
Cada escrita é efectuada no servidor primário e no secundário, mas esta sequência é efectuada de forma transparente ao cliente. 
\end{enumerate}
