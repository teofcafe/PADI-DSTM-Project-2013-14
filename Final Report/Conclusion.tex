\section{Conclusão}

A solução proposta de modo a tornar o \textbf{PADI-DSTM} uma aplicação sequencialmente consistente reside numa abordagem de TimeStamp Ordering\cite{ex1}. Nesta solução já estão contempladas tanto a tolerância a faltas como a migração de objectos extremamente acedidos, existindo distribuição de carga. Isto permite que o sistema se mantenha equilibrado, tanto a nível de computação como a nível de armazenamento. Parte das decisões tomadas para esta arquitectura (por exemplo, métricas que consideram um PadInt extremamente acedido), podem ser adaptados aos mais variados tipos de sistema, de modo a maximizar o desempenho. Analisando os resultados experimentais dos testes feitos ao nosso sistema, usando processadores intel i7, verificamos que o mesmo se mantém escalável e consistente, mesmo quando sobrecarregado. 
